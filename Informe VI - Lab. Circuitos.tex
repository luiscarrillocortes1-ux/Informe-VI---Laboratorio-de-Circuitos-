% main.tex
\documentclass[conference]{IEEEtran} % usa [journal] para artículos de revista

% ---- Paquetes recomendados ----
\usepackage[utf8]{inputenc}
\usepackage[T1]{fontenc}
\usepackage{lmodern}
% Comenta la siguiente línea si tu paper será en inglés
\usepackage[spanish,es-noshorthands]{babel}

\usepackage{graphicx}    % figuras
\usepackage{amsmath,amssymb}
\usepackage{siunitx}     % unidades
\usepackage{booktabs}    % tablas bonitas
\usepackage{url}         % URLs en referencias
\usepackage{cite}        % manejo de citas IEEE
\usepackage{microtype}   % mejor tipografía
\usepackage{float}

% ---- Información del artículo ----
\title{Título del Artículo en Formato IEEE}

\author{
\IEEEauthorblockN{Nombre Apellido\IEEEauthorrefmark{1}, Nombre Apellido\IEEEauthorrefmark{2}}
\IEEEauthorblockA{\IEEEauthorrefmark{1}Afiliación 1 \\
Email: autor1@ejemplo.com}
\IEEEauthorblockA{\IEEEauthorrefmark{2}Afiliación 2 \\
Email: autor2@ejemplo.com}
}

\begin{document}
\maketitle

\begin{abstract}
Este es el resumen (abstract). Debe ser conciso y describir brevemente el problema, el método y los resultados principales.
\end{abstract}


\section{Introducción}
Contexto del problema, motivación, contribuciones y estructura del documento~\cite{ieeehowto,lamport94}.

\section{Objetivos}
\begin{itemize}
  \item Implementar y simular un circuito serie con varias resistencias para calcular los valores de voltaje en cada una de ellas.
  \item Determinar la resistencia equivalente en un circuito serie.
  \item Comparar los valores te\'oricos, simulados e implementados de los voltajes y corriente en un circuito serial resistivo.
  \item Determinar la potencia disipada por cada resistor en un circuito serial.
  \item Identificar las partes principales del osciloscopio y comprender su funci\'on.
  \item Configurar adecuadamente los controles del osciloscopio para visualizar se\~nales.
  \item Medir caracter\'isticas de se\~nales el\'ectricas como amplitud, frecuencia, per\'iodo, fase y forma de onda utilizando el osciloscopio.
  \item Interpretar correctamente las formas de onda observadas.
  \item Familiarizarse con el funcionamiento de un generador de se\~nales.
\end{itemize}


\section{Desarrollo}
\subsection{Metodología I: Medición de voltaje}
Se calibra el osciloscopio para que la onda de sobre el eje x.
La fuente de voltaje fue ajustada a 5 V y sus cables se conectaron a la sonda del osciloscopio, la cual se colocó en el canal 1. Posteriormente, se configuró la perilla de control vertical en 2 V/div, observándose en la pantalla la señal correspondiente. Finalmente, la perilla de ajuste vertical fue variada con el fin de analizar los cambios en la forma de onda. Véase la Figura 1.
\begin{figure}[H]
    \centering
    \includegraphics[width=0.75\linewidth]{5V.JPG}
    \caption{Señal del osciloscopio conectado a una fuente de 5V}
    \label{fig:placeholder}
\end{figure}

\subsection{Metodología II: medición de voltaje AC}
Se monta el circuito que dice la guia experimental (veáse la figura 2), empleando una fuente de alimentación de 120 V a una frecuencia de 60 Hz. El multímetro se conectó en los puntos A y B, registrando en su pantalla la medición correspondiente.

\begin{figure}[H]
    \centering
    \includegraphics[width=0.75\linewidth]{Figura 4.jpg}
    \caption{Simulación circuito medición de voltaje a la salida de transformador}
    \label{fig:placeholder}
\end{figure}
En la figura 3, se puede observar como se toma el voltaje a la salida del transformador, lo que nos da 13.3 V en AC 
\begin{figure}[H]
    \centering
    \includegraphics[width=0.5\linewidth]{WhatsApp Image 2025-10-01 at 9.40.29 PM.jpeg}
    \caption{Voltaje obtenido}
    \label{fig:placeholder}
\end{figure}
Al desconectarlo y conectar ahora los nodos A y B en el osciloscopio nos dio que la señal obtenida fue: 
\begin{figure}[H]
    \centering
    \includegraphics[width=0.75\linewidth]{WhatsApp Image 2025-10-01 at 9.59.50 PM.jpeg}
    \caption{señal AC Osciloscopio}
    \label{fig:placeholder}
\end{figure}

Se registraron en la Tabla 1 los valores correspondientes al voltaje pico, voltaje pico a pico y voltaje efectivo-RMS.
\begin{table}[H]
    \centering
    \caption{Resultados medición señal AC Osciloscopio.}
    \label{tab:ejemplo}
    \begin{tabular}{|c|c|}
        \hline
        VOLTAJE PICO & 20.0 V  \\ \hline
        VOLTAJE PICO A PICO    & 72.0 V     \\ \hline
        VOLTAJE EFECTIVO-RMS   & 25.4 V     \\ \hline
    \end{tabular}
\end{table}



\section{Resultados}
\subsection{Figura de ejemplo}



\section{Discusión}
Analiza implicaciones, limitaciones y posibles mejoras.

\section{Conclusiones}
Resumen de hallazgos y trabajo futuro.

\section*{Agradecimientos}
(Optativo) Reconoce apoyos, proyectos o personas.

\bibliographystyle{IEEEtran}
\bibliography{refs} % archivo refs.bib

% \appendices
% \section{Prueba adicional}
% Contenido opcional de apéndices.

\end{document}
